
\chapter{章标题(一级标题)}

\section{节标题(二级标题)}

这里测试大段文字:

学位论文是学位授予单位研究生培养质量的重要标志,是研究生本人在学期间从事学术研究的成果体现,是对研究生综合能力的考核,也是研究生申请学位的主要依据。为保证我校研究生学位论文的质量,实现我校学位论文格式的规范化,参照GB/T 7713.1—2006《学位论文编写规则》,GB 7713-87《科学技术报告、学位论文和学术论文的编写格式》和GB/T 7714—2005《文后参考文献著录规则》等标准,特制定《天津大学关于博士、硕士学位论文统一格式的规定》。

\begin{enumerate}[label={(\arabic*)}]
    \item 测试列表条目
    \item 测试列表条目 
    \begin{enumerate}[label={(\roman*)}]
        \item 嵌套列表条目
        \item 嵌套列表条目
    \end{enumerate}
\end{enumerate}

这里测试插入图片:
\begin{figure}[!h]
    \centering\small
    \includegraphics[width=3cm]{tju_logo.png}
    \caption{天津大学校徽}
    \label{fig:logo}
\end{figure}
图~\ref{fig:logo} 所示为一张图片。

\subsection{小节标题(三级标题)}

普通的三线表效果如表~\ref{tab:1} 所示。
\begin{table}[thb]
    \centering\small
    \caption{一个普通的三线表\citep{mittelbach2004}}\label{tab:1}
    \begin{tabular}{L{3cm}C{3cm}R{3cm}}
        \toprule
        表头 & 表头 & 表头 \\
        \midrule
        表格内容 & 表格内容 & 表格内容 \\
        ... & ... & ... \\
        ... & ... & ... \\
        ... & ... & ... \\
        ... & ... & ... \\
        ... & ... & ... \\
        表格内容 & 表格内容 & 表格内容 \\
        \bottomrule
    \end{tabular}
\end{table}

如果表格需要注释,则需要使用 threeparttable 包,结果如表~\ref{tab:2} 所示。
\begin{table}[thb]
    \centering\small
    \caption{一个包含注释的三线表}\label{tab:2}
    \begin{threeparttable}
        \begin{tabular}{L{3cm}C{3cm}R{3cm}}
            \toprule
            表头 & 表头 & 表头 \\
            \midrule
            表格内容\tnote{*} & 表格内容 & 表格内容 \\
            表格内容\tnote{**} & 表格内容 & 表格内容 \\
            表格内容\tnote{***} & 表格内容 & 表格内容 \\
            ... & ... & ... \\
            ... & ... & ... \\
            ... & ... & ... \\
            ... & ... & ... \\
            ... & ... & ... \\
            \bottomrule
        \end{tabular}
        \begin{tablenotes}
            \item [*] 这是一个注释。
            \item [**] This is a table note.
            \item [***] 它们都在表格下方。
        \end{tablenotes} 
    \end{threeparttable}
\end{table}

如果表格需要跨页\footnotemark[1],则需要使用 longtable 包,结果如表~\ref{tab:3} 所示。

\footnotetext[1]{\LaTeX 可以动态调整图表位置,因此不建议使用跨页表格,除非表格真的超过整张页面。}

\begin{ThreePartTable}

% 表格内文字(及注释)的字体大小,不会影响表题
\begin{small}

% 表格注释
\begin{TableNotes}
    \item [a] 这是一个注释。
    \item [b] This is another table note.
    \item [c] 它们都在表格下方。
\end{TableNotes}

\begin{longtable}{L{3cm}C{3cm}R{3cm}}

    % 以下定义表格顶部格式
    \caption{一个跨页包含注释的三线表}\label{tab:3} \\    
    \toprule
    表头 & 表头 & 表头 \\
    \midrule
    \endfirsthead

    % 以下定义除了第一页以外的表格顶部格式
    (续) \\
    \toprule
    表头 & 表头 & 表头 \\
    \midrule
    \endhead

    % 以下定义表格底部格式
    \bottomrule
    \endfoot

    % 以下定义最后一页的表格底部格式(有注释)
    \bottomrule
    \insertTableNotes
    \endlastfoot

    % 以下为表格内容
    表格内容\tnote{a} & 表格内容 & 表格内容 \\
    表格内容\tnote{b} & 表格内容 & 表格内容 \\
    表格内容\tnote{c} & 表格内容 & 表格内容 \\
    ... & ... & ... \\
    ... & ... & ... \\
    ... & ... & ... \\
    ... & ... & ... \\
    ... & ... & ... \\
    ... & ... & ... \\
    ... & ... & ... \\
    ... & ... & ... \\
    ... & ... & ... \\
    ... & ... & ... \\
    ... & ... & ... \\
    ... & ... & ... \\
    表格内容 & 表格内容 & 表格内容 \\
    表格内容 & 表格内容 & 表格内容 \\
    表格内容 & 表格内容 & 表格内容 \\

\end{longtable}
\end{small}
\end{ThreePartTable}


\subsubsection{四级标题}

这是一个参考文献引用测试:\citet{mittelbach2004,张琪昌2005} 提出了:
\begin{equation}\label{eq:1}
    a + b = c
\end{equation}
其中 $c$ 为一个东西 \citep{ctex2022,kottwitz2015,张琪昌2005}。值得注意的是,公式~\ref{eq:1} 里面的$a$和$b$也是两个东西。

\chapter{另一个章标题(一级标题)}

测试新一章是否正确地从奇数页开始。