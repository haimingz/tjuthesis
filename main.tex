\documentclass[openright,twoside]{book}

\usepackage{tjustyle}

% 根据需求导入宏包,自定义命令。以下内容如不需要都可删除。

\RequirePackage{threeparttable} % 用于给表格添加注释
\RequirePackage{threeparttablex} % 用于给长表格添加注释
\RequirePackage{longtable} % 用于绘制长表格

% 定义表格列宽的命令
\newcolumntype{L}[1]{>{\raggedright\let\newline\\\arraybackslash\hspace{0pt}}m{#1}}
\newcolumntype{C}[1]{>{\centering\let\newline\\\arraybackslash\hspace{0pt}}m{#1}}
\newcolumntype{R}[1]{>{\raggedleft\let\newline\\\arraybackslash\hspace{0pt}}m{#1}}


\graphicspath{{figures/}} % 指定图片保存地址

\begin{document}

% 扉页

\pagestyle{empty} % 去除页眉页脚

\begin{center} % 本页内容居中

\vspace*{0.8cm} % 根据需要调整距离

\includegraphics[height=1.75cm,keepaspectratio]{figures/tjulogo.png} % 设置校徽 (天津大学关于博士、硕士学位论文统一格式的规定(2021年修订20250605更新扉页版))

\vspace*{0.8cm} % 根据需要调整距离

\songti\zihao{2}\textbf{中文题目(宋体二号加粗)}

\vspace{0.8cm} % 根据需要调整距离

\zihao{4}\textbf{English Title(Times New Roman, 14-22 pt., Bold)}

\vfill

\zihao{-4} % 切换为小四字号绘制下面两个表格

% 基本信息表
\begin{table}[h] 
    \centering 
    \songti
    \renewcommand*{\arraystretch}{1.35} % 设置行高
    \begin{tabular}{cc}
        \textbf{一级学科:} & \textbf{\underline{\makebox[14em][c]{输入}}} \\
        \textbf{研究方向:} & \textbf{\underline{\makebox[14em][c]{输入}}} \\
        \textbf{作者姓名:} & \textbf{\underline{\makebox[14em][c]{输入}}} \\
        \textbf{指导教师:} & \textbf{\underline{\makebox[14em][c]{输入}}} \\
    \end{tabular}
\end{table}

\vspace{0.5cm}  % 根据需要调整距离

% 答辩委员会名单表,可根据实际答辩委员数增加行数,注意同时修改 \multirow 的第一个参数
\begin{table}[h] 
    \centering 
    \songti
    \renewcommand*{\arraystretch}{1.35} % 设置行高
    \begin{tabular}{|C{0.22\textwidth}| C{0.16\textwidth}| C{0.14\textwidth}| C{0.36\textwidth}|}
    \hline
    \textbf{答辩日期}              & \multicolumn{3}{c|}{20XX年XX月XX日} \\ \hline
    \textbf{答辩委员会}            &  \textbf{姓名} &  \textbf{职称} &  \textbf{工作单位}  \\ \hline
    \textbf{主席}                  &                &                &                     \\ \hline
    \multirow{6}{*}{\textbf{委员}} &                &                &                     \\ \cline{2-4} 
                                   &                &                &                     \\ \cline{2-4}
                                   &                &                &                     \\ \cline{2-4}
                                   &                &                &                     \\ \cline{2-4}
                                   &                &                &                     \\ \cline{2-4}
                                   &                &                &                     \\ \hline
    \end{tabular}
\end{table}

\vspace{0.5cm}  % 根据需要调整距离

\zihao{4}{天津大学XXXX学院\\二〇XX年XX月}

\vspace{0.5cm}  % 根据需要调整距离

\end{center} % 停止居中
 % 扉页
\clearpage{\cleardoublepage}
\thispagestyle{empty}

\vspace*{1cm}

\begin{center}
    \zihao{-2}{独创性声明}    
\end{center}

\vspace{1cm}

\zihao{-4}
本人声明所呈交的学位论文是本人在导师指导下进行的研究工作和取得的研究成果,除了文中特别加以标注和致谢之处外,论文中不包含其他人已经发表或撰写过的研究成果,也不包含为获得{\underline{\zihao{4}\textbf{\kaishu{~~天津大学~~}}}}或其他教育机构的学位或证书而使用过的材料。与我一同工作的同志对本研究所做的任何贡献均已在论文中作了明确的说明并表示了谢意。

\vspace{1cm} 

\noindent 学位论文作者签名:\makebox[4cm][s]{} 签字日期:\makebox[2cm][s]{}年\makebox[1cm][s]{}月\makebox[1cm][s]{}日

\vspace{2cm}

\begin{center}
    \zihao{-2}{学位论文版权使用授权书}    
\end{center}

\vspace{1cm}

\zihao{-4}
本学位论文作者完全了解{\underline{\zihao{4}{\textbf{\kaishu{~~天津大学~~}}}}}有关保留、使用学位论文的规定。特授权{\underline{\zihao{4}{\textbf{\kaishu{~~天津大学~~}}}}}可以将学位论文的全部或部分内容编入有关数据库进行检索,并采用影印、缩印或扫描等复制手段保存、汇编以供查阅和借阅。同意学校向国家有关部门或机构送交论文的复印件和磁盘。

(保密的学位论文在解密后适用本授权说明)

\vspace{1cm}

\noindent 学位论文作者签名:\makebox[4cm][s]{} 导师签名:\makebox[4cm][s]{}

\vspace{0.5cm}

\noindent 签字日期:\makebox[2cm][s]{}年\makebox[1cm][s]{}月\makebox[1cm][s]{}日 
\quad 签字日期:\makebox[2cm][s]{}年\makebox[1cm][s]{}月\makebox[1cm][s]{}日
 % 独创性声明

\frontmatter % 开始有页码(大写罗马)

\chapter*{\songti\zihao{2}\textbf{摘 \qquad 要}}
\thispagestyle{noheader}

中文摘要。

\

\noindent{\zihao{4}\textbf{关键词:}} 关键词,关键词,关键词,关键词,关键词

\chapter*{\songti\zihao{2}\textbf{ABSTRACT}}
\thispagestyle{noheader}

Abstract in English.

\

\noindent{\zihao{4}\textbf{KEY WORDS: }} Keyword, Keyword, Keyword, Keyword, Keyword
 % 中英文摘要

\tableofcontents % 目录

\mainmatter % 进入正文部分,页码变为阿拉伯数字


\chapter{章标题(一级标题)}

\section{大段文字}

测试大段文字。\zhlipsum[1-2] % dummy text

\begin{enumerate}[(1)]
    \item 测试列表条目 \zhlipsum[1]
    \item 测试列表条目 \zhlipsum[1]
\end{enumerate}

\subsection{图表}

这里测试插入图片:
\begin{figure}[!h]
    \centering\small
    \includegraphics[width=8cm]{example-image}
    \caption{示例图片}
    \label{fig1}
\end{figure}
\cref{fig1}所示为一张图片。

这里测试插入表格:
\begin{table}[!htb]
    \centering\small
    \caption{测试表格\citep{smith2024}}
    \begin{tabular}{ccc}
        \toprule
        序号 & 姓名 & 年龄 \\
        \midrule
        1  & 张三 & 20 \\
        2  & 李四 & 22 \\
        3  & 王五 & 21 \\
        \bottomrule
    \end{tabular}
    \label{tab:test}
\end{table}
\cref{tab:test}所示为一张表格。

\subsection{公式}

测试公式,如\cref{eq1}所示:
\begin{equation}\label{eq1}
    f(x) = \int_{-\infty}^\infty  \hat f(x)\xi\,e^{2 \pi i \xi x}  \,\mathrm{d}\xi 
\end{equation}
\begin{equation}
    A_{m\times n}=  
        \begin{bmatrix}  
            a_{11}& a_{12}& \cdots  & a_{1n} \\  
            a_{21}& a_{22}& \cdots  & a_{2n} \\  
            \vdots & \vdots & \ddots & \vdots \\  
            a_{m1}& a_{m2}& \cdots  & a_{mn}  
        \end{bmatrix}  
    =\left [ a_{ij}\right ] 
\end{equation}

\subsection{引用}

测试引用参考文献:\citet{zhang2023} 证明 \citet{lee2025} 的结论是正确的\citep{lisi2022}。
测试引用多个参考文献:\citet{brown2026,zhao2024,wilson2023} 的结论是正确的\citep{zhou2025,zhao2024}。
\citet{lisi2022}的研究表明\cite{zhang2023,lee2025,zhou2025}。

\chapter{另一个章标题(一级标题)}
\label{ch:2}

测试新一章是否正确地从奇数页开始。

\section{复杂引用(基于Cleveref)}
\label{sec:2.1}

这里测试第二章图片标号:
\begin{figure}[!h]
    \centering\small
    \includegraphics[width=5cm]{example-image}
    \caption{示例图片}
    \label{fig2}
\end{figure}

\begin{figure}[!h]
    \centering\small
    \includegraphics[width=5cm]{example-image}
    \caption{示例图片}
    \label{fig3}
\end{figure}

测试引用多张图片:\cref{fig1,fig2,fig3}。

测试引用图片、公式和表格:\cref{fig1,eq1,tab:test}。

测试引用章节:见\cref{ch:2,sec:2.1}。

\subsection{脚注}

这里测试脚注\footnote{这是一个脚注示例。}的效果。
这是另一个脚注\footnote{这是第二个脚注示例。}。 % 导入正文示例

\bibliographystyle{gbt7714} % 指定参考文献格式文件
\bibliography{references} % 指定参考文献数据文件

\appendix
\chapter{附录标题}

附录内容。 % 添加附录

\backmatter % 以后的章节没有编号

\chapter{发表论文和参加科研情况说明}

发表论文和参加科研情况说明。
\begin{enumerate}[label={[\arabic*]}]
    \item 我的一篇论文
    \item 我的一篇论文
    \item 我的一篇论文
\end{enumerate} % 发表论文和参加科研情况说明
\chapter{\texorpdfstring{致 \qquad 谢 }{致 谢}}

致谢内容。 % 致谢

\end{document}
